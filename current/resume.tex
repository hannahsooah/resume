% !TEX TS-program = xelatex

\documentclass{article}
\usepackage[margin=0.5in]{geometry}
\usepackage{sectsty, xcolor, hyperref, enumitem, titlesec, titling, fontspec}

\newfontfamily{\FAFR}{Font Awesome 5 Free Regular}
\def\faEmail{{\FAFR \symbol{"F0E0}}} % Email
\def\faPhone{{\FAFR \symbol{"F095}}} % Phone
\newfontfamily{\FAB}{Font Awesome 5 Brands Regular}
\def\faLinkedin{{\FAB \symbol{"F08C}}} % Linkedin
\def\faGithub{{\FAB \symbol{"F09B}}} % Github

\setmainfont{Poppins}
\pagenumbering{gobble}
\linespread{1.25}

\definecolor{titlecolor}{RGB}{56,140,236} % #388cec
\sectionfont{\color{titlecolor}}


\titlespacing*{\section}{0pt}{5pt}{2pt}
\titlespacing*{\subsection}{0pt}{3pt}{0pt}

\titleformat{\section}
{\bfseries\color{titlecolor}\large}{}{0em}{}
\titleformat{\subsection}[runin]
{\bfseries}{}{0em}{}

\setlist{leftmargin=1.5em}
\renewcommand\labelitemi{-}
\let\olditemize=\itemize \let\endolditemize=\enditemize
\renewenvironment{itemize}{\olditemize[topsep=0em] \itemsep-.3em}{\endolditemize}

\newcommand{\link}[1]{\href{https://#1}{#1}}
\newcommand{\entry}[3]{\quad\textbf{#1}\\#2\qquad#3}

\renewcommand{\maketitle}{
  \begin{flushleft}
    \Huge\bfseries\theauthor
  \end{flushleft}
  \begin{bfseries}
    \faEmail    \hspace{1pt} \href{mailto:hannahso@berkeley.edu}{hannahso@berkeley.edu} \quad
    \faPhone    \hspace{1pt} (408) 593-7698 \quad
    \faLinkedin \hspace{1pt} \href{https://www.linkedin.com/in/hannahso}{hannahso} \quad
    \faGithub   \hspace{1pt} \href{https://github.com/hannahsooah/}{hannahsooah}
  \end{bfseries}
}


\begin{document}
\author{\color{titlecolor}Hannah S. Oh}
\maketitle


\section{Education}
\subsection{University of California, Berkeley}
\entry{Electrical Engineering and Computer Science, BS}{Aug 2020 - May 2023}\\
Coursework:
  Algorithms,
  Data Structures,
  Machine Structures,
  Database Systems,
  AI, 
  Linux SysAdmin DeCal,
  Discrete Mathematics,
  Designing Information Devices and Systems


\section{Experience}

\subsection{Production Engineer Intern}
\entry{Facebook}{Jun 2021 - Aug 2021}{Menlo Park, CA (Remote)}
\begin{itemize}
  \item Implemented relative computing resource units for future use in server inventory management systems
  \item Refactored scheduler to be cross-compatible with capacity requests of different resource types
  \item Created alerts and detectors to monitor invalid resource configurations
\end{itemize}

\subsection{Academic Course Staff}
\entry{UC Berkeley EECS Department}{Jan 2021 - Current}{Berkeley, CA}
\begin{itemize}
  \item Reinforce student knowledge of circuit analysis, signals, and controls as a member of course staff
  \item Staff weekly reviews for students; focus on conceptual application through projects, homework, and labs
\end{itemize}


\section{Projects}

\subsection{NumC}\quad{April 2021}
\begin{itemize}
  \item Developed a Python array-processing API written in C mimicking NumPy functionalities
  \item Implemented multi-threading, SIMD, loop unrolling, and blocking to optimize matrix arithmetic
  \item Achieved 90x speedup for matrix multiplication and 1000x speedup for matrix powering
\end{itemize}

\subsection{Gitlet}\quad{March 2021}
\begin{itemize}
  \item Built a custom version control system in Java from scratch, mimicking git
  \item Wrote a breadth-first search algorithm to detect split points in commit history for correct branch merging
  \item Functionality includes init, add, commit, rm, log, global-log, find, status, checkout, branch, merge, reset, etc.
\end{itemize}

\subsection{Classify}\quad{Feb 2021}
\begin{itemize}
  \item Wrote RISC-V assembly code to run a simple Artificial Neural Network (ANN) on a RISC-V simulator
  \item Implemented basic operations including vector dot product and matrix multiplication using assembly code
  \item Loaded a pretrained ANN and executed it to classify handwritten digits from the MNIST benchmark set
\end{itemize}

\subsection{Lisp Dialect Interpreter}\quad{Nov 2020}
\begin{itemize}
  \item Wrote a recursive interpreter and tokenizer for a List dialect using Python
  \item Created the REPL to evaluate expressions and implemented tail recursion to optimize recursion
\end{itemize}

\subsection{Autocorrected Typing Software}\quad{Sept 2020}
\begin{itemize}
  \item Measures typing speed and accuracy and has autocorrect feature to correct user spelling mistakes 
  \item Added multiplayer compatibilities to race friends and professors
\end{itemize}

\section{Awards}
{Presidential Scholars Program Nominee, Jan 2020}\quad\quad{National Merit Scholarship, Feb 2020} \\
{The President’s Volunteer Service Award, Nov 2019}\quad\quad{FUHSD Community Service Award, May 2020}

\section{Skills}
Technical:
  Rust, Java, C, Python, SQL, LaTeX, Lisp \\
Tools:
  Vim, Git, Mercurial, Bash, Eclipse, IntelliJ \\
Languages:
  English (native), Korean (native), Spanish (elementary)


\end{document}
