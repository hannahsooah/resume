% !TEX TS-program = xelatex

\documentclass{article}
\usepackage[margin=0.5in]{geometry}
\usepackage{sectsty, xcolor, hyperref, enumitem, titlesec, titling, fontspec}

\newfontfamily{\FAFR}{Font Awesome 5 Free Regular}
\def\faEmail{{\FAFR \symbol{"F0E0}}} % Email
\def\faPhone{{\FAFR \symbol{"F095}}} % Phone
\newfontfamily{\FAB}{Font Awesome 5 Brands Regular}
\def\faLinkedin{{\FAB \symbol{"F08C}}} % Linkedin
\def\faGithub{{\FAB \symbol{"F09B}}} % Github

\setmainfont{Poppins}
\pagenumbering{gobble}
\linespread{1.1}

\definecolor{titlecolor}{RGB}{56,140,236} % #388cec
\sectionfont{\color{titlecolor}}


\titlespacing*{\section}{0pt}{5pt}{2pt}
\titlespacing*{\subsection}{0pt}{3pt}{0pt}

\titleformat{\section}
{\bfseries\color{titlecolor}\large}{}{0em}{}
\titleformat{\subsection}[runin]
{\bfseries}{}{0em}{}

\setlist{leftmargin=1.5em}
\renewcommand\labelitemi{-}
\let\olditemize=\itemize \let\endolditemize=\enditemize
\renewenvironment{itemize}{\olditemize[topsep=0em] \itemsep-.3em}{\endolditemize}

\newcommand{\link}[1]{\href{https://#1}{#1}}
\newcommand{\entry}[3]{\quad\textbf{|\quad#1}\\#2\qquad#3}

\renewcommand{\maketitle}{
  \begin{flushleft}
    \Huge\bfseries\theauthor
  \end{flushleft}
  \begin{bfseries}
    \faEmail    \hspace{1pt} \href{mailto:hannahso@berkeley.edu}{hannahso@berkeley.edu} \quad
    \faPhone    \hspace{1pt} (408) 593-7698 \quad
    \faLinkedin \hspace{1pt} \href{https://www.linkedin.com/in/hannahso}{hannahso} \quad
    \faGithub   \hspace{1pt} \href{https://github.com/hannahsooah/}{hannahsooah}
  \end{bfseries}
}


\begin{document}
\author{\color{titlecolor}Hannah S. Oh}
\maketitle


\section{Education}
\subsection{University of California, Berkeley}\quad\textbf{Electrical Engineering and Computer Science, BS}\\
{Aug 2020 - Dec 2023}\\
Coursework:
  Operating Systems,
  Algorithms,
  Data Structures,
  Internet Protocols,
  Computer Security,
  Communication Networks,
  Machine Structures,
  Database Systems,
  AI, 
  Linux SysAdmin
  %Discrete Mathematics,
  %Designing Information Devices and Systems


\section{Experience}

\subsection{Software Engineer Intern}
\entry{Meta}{May 2022 - Aug 2022}{New York City, NY}
\begin{itemize}
  \item Built a debug adapter for command line debuggers to communicate using Debug Adapter Protocol
  \item Refactored a VSCode extension to launch record-and-replay debug sessions in the IDE's debugging UI
  \item Multithreaded the server to enhance performance and manage multiple streams of input independently
  \item Integrated the product with existing tooling and enabled multiple launch methods for various use cases
\end{itemize}

\subsection{Production Engineer Intern}
\entry{Meta}{Jun 2021 - Aug 2021}{Menlo Park, CA (Remote)}
\begin{itemize}
  \item Implemented relative computing resource units for future use in server inventory management systems
  \item Refactored scheduler to be cross-compatible with capacity requests of different resource types
  \item Enabled logging and created alerts and detectors to monitor invalid resource configurations
\end{itemize}

\subsection{Academic Course Staff}
\entry{UC Berkeley EECS Department}{Jan 2021 - May 2022}{Berkeley, CA}
\begin{itemize}
  \item Led weekly reviews to reinforce student knowledge of machine architecture, high-level language support, and operating systems (I/O, interrupts, memory management, process switching)
  \item Aided students in conceptual application through projects, homework, and labs
\end{itemize}

%\section{Management Experience}

%\subsection{Committee Director}
%\entry{ASUC Office of Senator Kim}{April 2022 - Current}{Berkeley, CA}
%\begin{itemize}
  %\item Lead meetings for hosting on-campus speaker events and compile resources for rotating topics
  %\item Represent students' public opinion on the topic at hand`and advocate for appropriate official actions
%\end{itemize}
%
%\subsection{Vice President}
%\entry{OTTA United, a 501(c)(3) Organization}{August 2019 - May 2021}{Sunnyvale, CA}
%\begin{itemize}
  %\item Managed organizational planning for utilization and delegation of manpower and financial resources
  %\item Applied for funding and grants as deemed necessary from the budget and major organizational goals
  %\item Organization of hundreds of volunteers to produce face shields and face masks for Bay Area hospitals
%\end{itemize}
%

\section{Projects}

\subsection{Addepar EntitySearch}\quad{Sept 2021}
\begin{itemize}
  \item Created an automatic cloud deployment of AWS OpenSearch, using its REST API for queries and uploads
  \item Developed a scalable search endpoint used in workflows in AMP for clients to handle large number of nodes
  \item Implemented a front end to input EntitySearch queries and display results using React and Springboot
\end{itemize}

\subsection{NumC}\quad{Apr 2021}
\begin{itemize}
  \item Developed a Python array-processing API written in C mimicking NumPy functionalities
  \item Implemented multi-threading, SIMD, loop unrolling, and blocking to optimize matrix arithmetic
  \item Achieved 90x speedup for matrix multiplication and 1000x speedup for matrix powering
\end{itemize}

\subsection{Gitlet}\quad{Mar 2021}
\begin{itemize}
  \item Built a custom version control system in Java from scratch, mimicking git
  \item Created a design doc with data structures, algorithms, and persistence for clear documentation
  \item Wrote a breadth-first search algorithm to detect split points in commit history for correct branch merging
  \item Functionality includes init, add, commit, rm, log, global-log, find, status, checkout, branch, merge, reset, etc.
\end{itemize}

\subsection{Classify}\quad{Feb 2021}
\begin{itemize}
  \item Wrote RISC-V assembly code to run a simple Artificial Neural Network (ANN) on a RISC-V simulator
  \item Implemented basic operations including vector dot product and matrix multiplication using assembly code
  \item Loaded a pretrained ANN and executed it to classify handwritten digits from the MNIST benchmark set
\end{itemize}
\subsection{Lisp Dialect Interpreter}\quad{Nov 2020}
\begin{itemize}
  \item Wrote a recursive interpreter and tokenizer for a List dialect using Python
  \item Created the REPL to evaluate expressions and implemented tail recursion to optimize recursion
\end{itemize}

%\subsection{Autocorrected Typing Software}\quad{Sept 2020}
%\begin{itemize}
  %\item Measures typing speed and accuracy and has autocorrect feature to correct user spelling mistakes 
  %\item Added multiplayer compatibilities to race friends and professors
%\end{itemize}

%\section{Awards}
%{Presidential Scholars Program Nominee, Jan 2020}\quad\quad{National Merit Scholarship, Feb 2020} \\
%{The President’s Volunteer Service Award, Nov 2019}\quad\quad{FUHSD Community Service Award, May 2020}

\section{Skills}
Languages:
  Rust, Java, C, Python, RISC-V, SQL, LaTeX, Go, C++, Lisp, React\\
Tools:
  Neovim, Git, Mercurial, Make, Terraform, VSCode, IntelliJ, Bash, Eclipse\\
%Languages:
  %English (native), Korean (native), Spanish (elementary)


\end{document}
